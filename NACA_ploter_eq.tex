\documentclass[11pt, a4paper]{article}

\usepackage{amsmath, amssymb, titling}
\usepackage[margin=2.5cm]{geometry}
\usepackage[colorlinks=true, linkcolor=black, urlcolor=black, citecolor=black]{hyperref}
\usepackage{url}
\usepackage{graphicx}
\usepackage{caption}
\usepackage{subcaption}
\usepackage{float}
\usepackage{cancel}
\usepackage{fancyhdr, lastpage}
\usepackage{fourier-orns}
\usepackage{xcolor}
\usepackage{nomencl}
\makenomenclature
\usepackage{etoolbox}
\usepackage{sidecap}
\usepackage{adjustbox}
\usepackage{listings}
\usepackage{matlab-prettifier}
\usepackage[T1]{fontenc}

\sidecaptionvpos{figure}{c}
\setlength{\headheight}{18.2pt}
\setlength{\nomlabelwidth}{1.5cm}

\renewcommand\maketitlehooka{\null\mbox{}\vfill}
\renewcommand\maketitlehookd{\vfill\null}

\renewcommand{\headrule}{\vspace{-5pt}\hrulefill\raisebox{-2.1pt}{\quad\leafleft\decoone\leafright\quad}\hrulefill}
\newcommand{\parder}[2]{\frac{\partial {#1}}{\partial {#2}}}
% \renewcommand\nomgroup[1]{%
%   \item[\bfseries
%   \ifstrequal{#1}{F}{Far--Away Properties}{%
%   \ifstrequal{#1}{N}{Dimensionless Numbers}{%
%   \ifstrequal{#1}{M}{Matrices}{%
%   \ifstrequal{#1}{D}{Diagonals}{%
%   \ifstrequal{#1}{V}{Vectors}{%
%   \ifstrequal{#1}{P}{Dimensionless Average Properties}{}}}}}}
% ]}

\title{Intro to Turbulent Flow \\ HW1}
\author{Almog Dobrescu ID 214254252}

% \pagestyle{fancy}
\cfoot{Page \thepage\ of \pageref{LastPage}}

\begin{document}

% \thispagestyle{empty}
% \maketitle
% \newpage

\pagenumbering{roman}

\vfil
\tableofcontents
\vfil
% \listoffigures
% \vfil
\lstlistoflistings
\newpage

% \printnomenclature
% \newpage

\pagestyle{fancy}
\pagenumbering{arabic}
\setcounter{page}{1}

\section{4 digits - MPxx}
\begin{itemize}
    \item $m=M/100$
    \item $p=P/10=x_{mc}$
    \item $t=xx$ - The maximum thickness as chord percentage, so $t=12$ means maximum thickness is $12\%$ of the chord.
    \item The leading edge approximates a cylinder with a radius of $r=1.1019t^2$.
\end{itemize}
\begin{equation}
    y_c=\left\{\begin{array}{cc}
        \displaystyle \frac{m}{p^2}\left(2px-x^2\right) & 0\le x<p \\\\
        \displaystyle \frac{m}{\left(1-p\right)^2}\left(\left(1-2p\right)+2px-x^2\right) & p\le x\le 1 \\\\
    \end{array}\right.
\end{equation}
\begin{equation}
    \frac{dy_c}{dx}=\left\{\begin{array}{cc}
        \displaystyle \frac{2m}{p^2}\left(p-x\right) & 0\le x<p \\\\
        \displaystyle \frac{2m}{\left(1-p\right)^2}\left(p-x\right) & p\le x\le 1 \\\\
    \end{array}\right.
\end{equation}

\subsection{Surfaces Generation}
Open airfoil:
\begin{equation}
    y_t=5t\left(0.969\sqrt{x}-0.1260x-0.3516x^2+0.2843x^3-0.1015x^4\right)
\end{equation}
To close the airfoil, the sum of the coefficient needs to be equal to 1. Changing the last coefficient results in the smallest change to the overall shape of the airfoil.
\begin{equation}
    y_t=5t\left(0.969\sqrt{x}-0.1260x-0.3516x^2+0.2843x^3-0.1036x^4\right)
\end{equation}
The respectively upper and lower airfoil surface are given by:
\begin{equation}
    \begin{matrix}
        \begin{array}{rcl}
            x_U & = & x-y_t\sin\left(\theta\right) \\
            x_L & = & x+y_t\sin\left(\theta\right)
        \end{array} & \begin{array}{rcl}
            y_U & = & y_c+y_t\cos\left(\theta\right) \\
            y_L & = & y_c-y_t\cos\left(\theta\right)
        \end{array}
    \end{matrix}
\end{equation}
Where:
\begin{itemize}
    \item $\displaystyle \theta=\arctan\left(\frac{dy_c}{dx}\right)$
\end{itemize}

\newpage
\section{5 digits - LPSxx}
\begin{itemize}
    \item $\displaystyle x_{mc}=0.05P$.
    \item $\displaystyle CL_i=0.15L$.
    \item $t=xx$ - The maximum thickness as chord percentage, so $t=12$ means maximum thickness is $12\%$ of the chord.
    \item The S digit is only between 1 and 5 (because the polynomial approximations).
    \item The parameters are calculated at $L=2$, and linearly scaled for a different desired design lift coefficient $CL_i$ as explained at Sec.\ref{sec: scaling 5 digits}.
\end{itemize}

\subsection{$S=0$}
\begin{equation}
    y_c=\left\{\begin{array}{cc}
        \displaystyle \frac{k_1}{6}\left(x^3-3rx^2+r^2\left(3-r\right)x\right) & 0\le x<r \\\\
        \displaystyle \frac{k_1r^3}{6}\left(1-x\right) & r\le x\le 1 \\\\
    \end{array}\right.
\end{equation}
\begin{equation}
    \frac{dy_c}{dx}=\left\{\begin{array}{cc}
        \displaystyle \frac{k_1}{6}\left(3x^2-6rx+r^2\left(3-r\right)\right) & 0\le x<r \\\\
        \displaystyle -\frac{k_1r^3}{6} & r\le x\le 1
    \end{array}\right.
\end{equation}

\subsubsection{$r$}
\begin{equation}
    \begin{array}{c}
        \displaystyle x_{mc}=r\left(1-\sqrt{\frac{r}{3}}\right) \\\\
        \displaystyle x_{mc}=r-\sqrt{\frac{r^3}{3}} \\\\
        \displaystyle \sqrt{\frac{r^3}{3}}=r-x_{mc} \\\\
        \displaystyle \frac{r^3}{3}=\left(r-x_{mc}\right)^2 \\\\
        \displaystyle \frac{r^3}{3}=r^2-2rx_{mc}+x_{mc}^2 \\\\
        \displaystyle \frac{1}{3}r^3-r^2+2rx_{mc}-x_{mc}^2=0 \\\\
        \vdots \\\\
        \text{solving numerically}
    \end{array}
\end{equation}

\subsubsection{$k_1$}
\begin{equation}
    k_1=\frac{6}{N}CL_i
\end{equation}
\begin{equation}
    N=\frac{3r-7r^2+8r^3-4r^4}{\sqrt{r-r^2}}-\frac{3}{2}\left(1-2r\right)\left(\frac{\pi}{2}-\arcsin\left(1-2r\right)\right)
\end{equation}

\subsubsection{fittings - L=2} \begin{table}[H]
    \renewcommand{\arraystretch}{1.5}
    \begin{center}
        \begin{tabular}{c|c|c|c}
            P & $x_mc$ & r & $k_1$ \\
            \hline
            1 & 0.05 & 0.0580 & 361.40 \\
            2 & 0.10 & 0.1260 & 51.649 \\
            3 & 0.15 & 0.2025 & 15.957 \\
            4 & 0.20 & 0.290  & 6.643 \\
            5 & 0.25 & 0.391  & 3.23
        \end{tabular}
    \end{center}
\end{table}
\begin{equation}
    \begin{array}{rcl}
        r & = & 3.333x_{mc}^3+0.7x_{cm}^2+1.197x_{cm}-0.004 \\\\
        k1 & = & 1.5149e6x_{mc}^4-1.0877e6x_{mc}^3+2.8646e5x_{mc}^2-3.2968e4x_{mc}+1.4202e3
    \end{array}
\end{equation}

\subsection{$S=1$}
\begin{equation}
    y_c=\left\{\begin{array}{cc}
        \displaystyle \frac{k_1}{6}\left((x-r)^3-\frac{k_2}{k_1}\left(1-r\right)^3x-r^3x+r^3\right) & 0\le x<r \\\\
        \displaystyle \frac{k_1}{6}\left(3(x-r)^2-\frac{k_2}{k_1}\left(1-r\right)^3-r^3\right) & r\le x\le 1 \\\\
    \end{array}\right.
\end{equation}
\begin{equation}
    \frac{dy_c}{dx}=\left\{\begin{array}{cc}
        \displaystyle \frac{k_1}{6}\left(\frac{k_2}{k_1}(x-r)^3-\frac{k_2}{k_1}\left(1-r\right)^3x-r^3x+r^3\right) & 0\le x<r \\\\
        \displaystyle \frac{k_1}{6}\left(3\frac{k_2}{k_1}(x-r)^2-\frac{k_2}{k_1}\left(1-r\right)^3-r^3\right) & r\le x\le 1
    \end{array}\right.
\end{equation}

\subsubsection{fittings - L=2} \begin{table}[H]
    \renewcommand{\arraystretch}{1.5}
    \begin{center}
        \begin{tabular}{c|c|c|c|c}
            P & $x_mc$ & r & $k_1$ & $\frac{k_2}{k_1}$ \\
            \hline
            2 & 0.10 & 0.130 & 51.999 & 0.000764 \\
            3 & 0.15 & 0.217 & 15.793 & 0.00677 \\
            4 & 0.20 & 0.318 & 6.520  & 0.0303 \\
            5 & 0.25 & 0.441 & 3.191  & 0.1355
        \end{tabular}
    \end{center}
\end{table}
\begin{equation}
    \begin{array}{rcl}
        r & = & 10.6667x_{mc}^3-2x_{cm}^2+1.7333{cm}-0.034 \\\\
        k_1 & = & -2.7973e4x_{mc}^3+1.7973e4x_{cm}^2-3.8884e3x_{cm}+289.076 \\\\ 
        \displaystyle \frac{k_2}{k_1} & = & 85.528x_{mc}^3-34.9828x_{cm}^2+4.8032x_{cm}-0.2153
    \end{array}
\end{equation}

\subsection{Scaling for Different L digit}
\label{sec: scaling 5 digits}
In order to scale for L values different from 2, just multiply the $y_c$ values:
\begin{equation}
    \begin{array}{c}
        \displaystyle {y_c}_\text{scaled}=\frac{L}{2}y_c \\\\
        \displaystyle {\frac{dy_c}{dx}}_\text{scaled}=\frac{L}{2}\frac{dy_c}{dx}
    \end{array}
\end{equation}

\subsection{Surfaces Generation}
Open airfoil:
\begin{equation}
    y_t=5t\left(0.969\sqrt{x}-0.1260x-0.3516x^2+0.2843x^3-0.1015x^4\right)
\end{equation}
To close the airfoil, the sum of the coefficient needs to be equal to 1. Changing the last coefficient results in the smallest change to the overall shape of the airfoil.
\begin{equation}
    y_t=5t\left(0.969\sqrt{x}-0.1260x-0.3516x^2+0.2843x^3-0.1036x^4\right)
\end{equation}
The respectively upper and lower airfoil surface are given by:
\begin{equation}
    \begin{matrix}
        \begin{array}{rcl}
            x_U & = & x-y_t\sin\left(\theta\right) \\
            x_L & = & x+y_t\sin\left(\theta\right)
        \end{array} & \begin{array}{rcl}
            y_U & = & y_c+y_t\cos\left(\theta\right) \\
            y_L & = & y_c-y_t\cos\left(\theta\right)
        \end{array}
    \end{matrix}
\end{equation}
Where:
\begin{itemize}
    \item $\displaystyle \theta=\arctan\left(\frac{dy_c}{dx}\right)$
\end{itemize}

\vfil
\nocite{*}
\bibliographystyle{ieeetr}
\bibliography{references}

\appendix
\section{Code Examples}
\subsection{MatLab}
\begin{lstinputlisting}[captionpos=b,stringstyle=\color{magenta},frame=single, numbers=left, style=MatLab-editor, basicstyle=\mlttfamily\small, caption={Example code listing},mlshowsectionrules=true]{./NACA_4_or_5_digit_ploter.m}
\end{lstinputlisting}

\end{document}
